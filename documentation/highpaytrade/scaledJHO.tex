\documentclass[10pt, a4paper]{article}
\usepackage{latexsym}
\usepackage{amssymb,amsmath}
\usepackage[pdftex]{graphicx}
\usepackage{float}
\newcommand{\dbar}[1]{\Bar{\Bar{#1}}}

\usepackage{hyperref}
\usepackage{caption}
\captionsetup[table]{position=bottom} 

\topmargin = 0.1in \textwidth=5.7in \textheight=8.6in

\oddsidemargin = 0.1in \evensidemargin = 0.1in

% headers
\usepackage{fancyhdr}
\pagestyle{fancy}
\chead{} 
\rhead{\thepage} 
% footer
\lfoot{\small\scshape } 
\cfoot{} 
%%%% insert your name here %%%%
\rfoot{\footnotesize Michael Burton} 
\renewcommand{\headrulewidth}{.3pt} 
\renewcommand{\footrulewidth}{.3pt}
\setlength\voffset{-0.25in}
\setlength\textheight{648pt}

\begin{document}

\title{Scaling the JHO for Larger Payloads}
\author{Michael Burton}
\maketitle

This paper shows sizing study of a scaled version of the Jungle Hawk Owl (JHO) that would accomodate a 100 lbs payload drawing 650 W.  
The same optimization techniques used to size and generate performance characteristics of the JHO were used to do this study. 

\section*{Sizing Study}

Table~\ref{t:keyvals} shows the estimated sizing for a scaled version of the JHO to hold a 100 lbs payload drawing 650 W for 5 days.

\begin{table}[h!]
    \centering
    \begin{tabular}{lc}
        Variable         & Value         \\ \hline \hline
        MTOW             & 517 {[}lbs{]} \\ 
        Empty weight     & 262 {[}lbs{]} \\ 
        Span             & 44.5 {[}ft{]} \\ 
        Aircraft length  & 20 {[}ft{]}   \\ 
        Aspect ratio     & 26.6          \\ 
        Root chord       & 2.22 {[}ft{]} \\ 
        Engine weight    & 18 {[}lbs{]}  \\ 
        Max engine power & 16 {[}hp{]}   \\ 
        Max speed        & 147 {[}kts{]} \\ 
        Loiter speed     & 49 {[}kts{]}  \\ 
    \end{tabular}
    \caption{Key Sizing Parameters}
    \label{t:keyvals}
\end{table}

Figure~\ref{f:endurancetrade} shows how the aircraft size varies per the endurance requirement. 

\begin{figure}[H]
    \begin{center}
        \includegraphics[width=0.7\textwidth]{endurancetrade.pdf}
        \caption{\textbf{Trade of max takeoff weight vs endurance for scaled JHO carrying 100 lbs payload drawing 650 W.}}
        \label{f:endurancetrade}
    \end{center}
\end{figure}

\section*{Key Assumptions}

\begin{enumerate}

    \item Payload weight: 100 lbs
    \item Payload power draw: 650 W
    \item Mission profile:
        \begin{enumerate}
            \item Climb to 15,000 ft
            \item Cruise 200 nmi to station
            \item Loiter for 5 days at 15,000 ft
            \item Cruise 200 nmi to base
            \item Descend 
            \end{enumerate}
    \item 95th percentile world wide winds at 15,000 ft: 49 kts
    \item Configuration: 
        \begin{enumerate}
            \item Cylindrical fuselage with ellispodal and o-jive fairings
            \item Constant tapered wing
            \item Conventional tail
            \item Pusher engine placement
            \end{enumerate}
    \item ``Rubber engine''; size scales with required power
    \item Performance:
        \begin{enumerate}
            \item Propellor efficiency: 70\%
            \item Minimum BSFC: 0.316 kg/kW-hr
            \item Span efficiency: 90\%
            \end{enumerate}
    \item Margins:
        \begin{enumerate}
            \item Drag: 5\%
            \item Wing weight: 20\%
            \item Empennage weight: 10\%
            \item Fuselage weight: 50\%
            \end{enumerate}

\end{enumerate}

\section*{References}

For more information about Geometric Programming optimization see: \url{hoburg.mit.edu}

\end{document}
